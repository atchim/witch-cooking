%%%%%%%%%%%%
% Preamble %
%%%%%%%%%%%%

\documentclass
  [ aspectratio=169,
    english,
    hyperref={citecolor=blue,colorlinks=true,linkcolor=blue,urlcolor=blue},
    brazil]
  {beamer}

%%%%%%%%%%%%%
%% Packages %
%%%%%%%%%%%%%

%%%%%%%%%%%%%%
%%% Encoding %
%%%%%%%%%%%%%%

\usepackage[utf8]{inputenc}
\usepackage[T1]{fontenc}

%%%%%%%%%%%%%%%%%%%
%%% Miscellaneous %
%%%%%%%%%%%%%%%%%%%

\usepackage[portuguese]{babel}
\usepackage[backend=biber,indent,style=abnt]{biblatex}
\usepackage{csquotes}
\usepackage{palatino}

%%%%%%%%%%%%%%%%%%
%% Configuration %
%%%%%%%%%%%%%%%%%%

\usepackage{xspace}

\newcommand
  {\ifba}
  {%
    Instituto Federal de Educação, Ciência e Tecnologia da Bahia --- Campus
    Irecê\xspace}
\newcommand{\treesitter}{\textit{Tree-Sitter}\xspace}
\newcommand{\witchcooking}{\textit{Witch Cooking}\xspace}
\newcommand
  {\witchcookingsubtitle}
  {%
    Formatação Personalizada de Código-Fonte via a Linguagem de Consulta do
    \treesitter}

%%%%%%%%%%%%
%%% Beamer %
%%%%%%%%%%%%

\setbeamertemplate{frametitle continuation}{}
\setbeamertemplate{headline}{}
\setbeamertemplate{navigation symbols}{}

%%%%%%%%%%%%%%%%%%
%%% Bibliography %
%%%%%%%%%%%%%%%%%%

\addbibresource{../references.bib}

%%%%%%%%%%%%
%%% Colors %
%%%%%%%%%%%%

\definecolor{blue}{RGB}{41,5,195}

%%%%%%%%%%
%%% Font %
%%%%%%%%%%

\usefonttheme{professionalfonts}
\renewcommand\familydefault{\rmdefault}

%%%%%%%%%%%%%%%%
%%% Title Page %
%%%%%%%%%%%%%%%%

\title{\witchcooking}
\subtitle{\witchcookingsubtitle}
\author{Átila Gama Silva}
\date{\today}
\titlegraphic{\includegraphics[height=1cm]{../ifba-irece-logo.png}}

%%%%%%%%%%%%
% Document %
%%%%%%%%%%%%

\begin{document}
  %%%%%%%%%%%%%%
  % Title Page %
  %%%%%%%%%%%%%%

  \begin{frame}\titlepage\end{frame}

  %%%%%%%%%%%%%%%
  % Motivations %
  %%%%%%%%%%%%%%%

  \begin{frame}{Motivações}
    \begin{itemize}
      \item<2-> Surgiram das dificuldades ao estudar estilos de formatação de
                código-fonte em diversas linguagens de programação
                \begin{itemize}
                  \item<3-> Durante a análise dos estilos convencionais de
                            formatação
                            \begin{itemize}
                              \item<4-> Era imprescindível recorrer a
                                        diferentes \textit{prettyprinters}
                              \item<5-> Cada um com suas próprias configurações
                                        e níveis de suporte para esses estilos
                            \end{itemize}
                  \item<6-> Durante a análise dos estilos não convencionais
                            \begin{itemize}
                              \item<7-> A aplicação manual era inevitável%
                                        \onslide
                                          <8->
                                          {%
                                            , consumindo consideravelmente tempo
                                            e esforço}
                            \end{itemize}
                \end{itemize}
    \end{itemize}
  \end{frame}

  %%%%%%%%%%%%
  % Purposes %
  %%%%%%%%%%%%

  \begin{frame}{Propósitos}
    \begin{itemize}
      \item<2-> Desenvolver um software de linha de comando%
                \onslide<3->{ — de natureza prototípica —}%
                \onslide<4->{ para a formatação de código-fonte}%
                \onslide<5->{, tendo como objetivos}
                \begin{itemize}
                  \item<6-> Abranger uma gama de linguagens de programação
                  \item<7-> proporcionar a formatação personalizada via a
                            linguagem de consulta do \treesitter
                \end{itemize}
    \end{itemize}
  \end{frame}

  %%%%%%%%%%%%%%%%%%%%%%%
  % The Code Formatting %
  %%%%%%%%%%%%%%%%%%%%%%%

  \begin{frame}{A Formatação de Código-Fonte}
    \begin{itemize}
      \item<2-> Desde os primórdios da computação, métodos foram
                desenvolvidos para garantir que a saída impressa fosse
                formatada de maneira esteticamente agradável
                \apud{harris-1956-keyboard}[1]{yelland-2015-new}
      \item<3-> Esses métodos ganharam popularidade sob o termo
                \textit{``prettyprinting''}
      \item<4-> No desenvolvimento de software, o \textit{prettyprinting} é
                conhecido como formatação de código-fonte
    \end{itemize}
  \end{frame}

  \begin{frame}{A Formatação de Código-Fonte}
    \begin{itemize}
      \item<1-> Durante as décadas de 60 e 70, a linguagem de programação LISP
                proporcionou condições favoráveis para o avanço da formatação
                de código \cite[2]{yelland-2015-new}
      \item<2-> Em 1967, Bill Gosper desenvolveu o \textit{GRINDEF}%
                \onslide
                  <3->
                  {%
                    : considerado o primeiro \textit{prettyprinter} a mensurar
                    o tamanho das linhas e ter ciência de sua localização no
                    arquivo
                    \cites
                      {gosper-2023-twubblesome}
                      {griesemer-2022-cultural}}
      \item<4-> Posteriormente, \textcite{oppen-1980-prettyprinting} apresentou
                algoritmo inovador de formatação de código-fonte%
                \onslide
                  <5->
                  {%
                    : destacava-se por sua capacidade de formatar código
                    derivado de qualquer linguagem de programação}
                \begin{itemize}
                  \item<6-> O algoritmo necessitava que o código fosse anotado
                            com espaços em branco e delimitadores especiais
                            para marcar o início e fim de blocos
                  \item<7-> Assim, o código a ser fornecido ao algoritmo
                            precisaria ser processado por uma ferramenta
                            intermediária capaz de compreender a sintaxe da
                            linguagem e fornecer um código anotado de forma
                            adequada, permitindo que o algoritmo realizasse a
                            formatação apropriada
                \end{itemize}
    \end{itemize}
  \end{frame}

  \begin{frame}{A Formatação de Código-Fonte}
    \begin{itemize}
      \item<1-> Recentemente, \textcite{yelland-2015-new} descreveu um
                algoritmo que visa otimizar o layout do código em relação a uma
                noção intuitiva de custo de layout
                \begin{itemize}
                  \item<2-> Notavelmente, entre as abstrações de programação
                            empregadas para facilitar sua aplicação em diversas
                            linguagens e políticas de layout de código,
                            destacam-se os \textit{combinators}%
                            \onslide
                              <3->
                              {%
                                : funções geradoras que descrevem layouts
                                alternativos para o código-fonte}
                \end{itemize}
    \end{itemize}
  \end{frame}

  %%%%%%%%%%%%%%%%%%%%%%%%%%
  % The Tree-Sitter System %
  %%%%%%%%%%%%%%%%%%%%%%%%%%

  \begin{frame}{O Sitema \treesitter}
    \begin{itemize}
      \item<2-> O \treesitter \cite{tree-sitter-2023-tree} é um sistema
                multilíngue de análise sintática para ferramentas de
                programação%
                \onslide
                  <3->
                  {%
                    , desenvolvido como uma tentativa de solucionar problemas
                    presentes nas ferramentas de análise sintática da época}
                \onslide<4->{, tendo como objetivos}
                \begin{itemize}
                  \item<5-> Ser utilizado no ambiente de desenvolvimento para
                            produzir árvores de sintaxe a partir da análise de
                            códigos escritos em várias linguagens
                  \item<6-> Implementar a análise incremental, permitindo a
                            atualização da árvore de sintaxe em tempo real
                  \item<7-> Expor através da árvore de sintaxe os nós
                            representando suas construções gramaticais no
                            código (e.g., classes, funções, declarações, etc.)
                  \item<8-> Ser livre de dependências, assim beneficiando sua
                            adoção e aplicabilidade
                \end{itemize}
    \end{itemize}
  \end{frame}

  \begin{frame}{O Sitema \treesitter}
    \begin{itemize}
      \item<1-> Adicionalmente%
                \onslide
                  <2->
                  {%
                    , o \treesitter oferece uma pequena linguagem de consulta
                    declarativa que é capaz de expressar padrões da árvore
                    sintática por meio de \textit{S-expressions} e buscar
                    correspondências}
      \item<3-> Essa linguagem suporta operadores que permitem
                \begin{itemize}
                  \item<4-> A captura de nós
                  \item<5-> A quantificação de nós, análoga às expressões
                            regulares
                  \item<6-> O agrupamento de nós
                  \item<7-> As alternâncias de nós
                  \item<8-> O uso de \textit{wildcards}
                  \item<9-> A ordenação de nós
                \end{itemize}
    \end{itemize}
  \end{frame}

  %%%%%%%%%%%%%%%%
  % Bibliography %
  %%%%%%%%%%%%%%%%

  \begin{frame}[allowframebreaks]{Referências}\printbibliography\end{frame}
\end{document}