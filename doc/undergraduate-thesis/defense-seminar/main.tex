%%%%%%%%%%%%
% Preamble %
%%%%%%%%%%%%

\documentclass
  [ aspectratio=169,
    english,
    hyperref={citecolor=blue,colorlinks=true,linkcolor=blue,urlcolor=blue},
    brazil]
  {beamer}

%%%%%%%%%%%%%
%% Packages %
%%%%%%%%%%%%%

%%%%%%%%%%%%%%
%%% Encoding %
%%%%%%%%%%%%%%

\usepackage[utf8]{inputenc}
\usepackage[T1]{fontenc}

%%%%%%%%%%%%%%%%%%%
%%% Miscellaneous %
%%%%%%%%%%%%%%%%%%%

\usepackage[portuguese]{babel}
\usepackage[backend=biber,indent,style=abnt]{biblatex}
\usepackage[listings,minted,skins,xparse]{tcolorbox}
\usepackage{csquotes}
\usepackage{palatino}
\usepackage{tikz}
\usepackage{xspace}

%%%%%%%%%%%%
%% Aliases %
%%%%%%%%%%%%

\newcommand{\treesitter}{\textit{Tree-Sitter}\xspace}
\newcommand{\witchcooking}{\textit{Witch Cooking}\xspace}
\newcommand{\worktitle}{\witchcooking}
\newcommand
  {\worksubtitle}
  {%
    Formatação Multilíngue e Personalizada de Código-Fonte via o Sistema
    \treesitter}

%%%%%%%%%%%%%%%%%%
%% Configuration %
%%%%%%%%%%%%%%%%%%

%%%%%%%%%%%%
%%% Beamer %
%%%%%%%%%%%%

\setbeamertemplate{frametitle continuation}{}
\setbeamertemplate{headline}{}
\setbeamertemplate{navigation symbols}{}

%%%%%%%%%%%%%%%%%%
%%% Bibliography %
%%%%%%%%%%%%%%%%%%

\addbibresource{../references.bib}

%%%%%%%%%%%%%%%%%%%
%%% Code Snippets %
%%%%%%%%%%%%%%%%%%%

\renewcommand{\theFancyVerbLine}{\ttfamily\arabic{FancyVerbLine}}

\definecolor{codehighlight}{RGB}{240,240,170}
\newcommand{\hi}[1]{\colorbox{codehighlight}{#1}}
\setminted{highlightcolor={codehighlight},spacecolor=gray!50}

\NewTotalTCBox
  {\codesnippetinline}
  {O{text}vO{}}
  { bottom=0pt,
    enhanced,
    frame hidden,
    left=0pt,
    on line,
    right=0pt,
    sharp corners,
    top=0pt}
  {\mintinline[#3]{#1}{#2}}%

\NewTCBListing
  {codesnippet}
  {sO{text}m!O{}!O{}}
  { enhanced,
    center title,
    coltitle=black,
    fonttitle=\scriptsize,
    frame hidden,
    listing engine=minted,
    listing only,
    minted language=#2,
    minted options=
      { autogobble,
        breakanywhere,
        breakanywheresymbolpre={},
        breakautoindent=false,
        breaklines,
        breaksymbol={},
        fontsize=\scriptsize,
        linenos,
        xleftmargin=\IfBooleanTF{#1}{1.5}{1}em,
        #4},
    nobeforeafter,
    overlay=
      {%
        \begin{tcbclipinterior}
          \fill[gray!25]
          (frame.south west)
          rectangle
          ([xshift=\IfBooleanTF{#1}{2.5}{2}em]frame.north west);
        \end{tcbclipinterior}},
    sharp corners,
    title={#3},
    #5}

%%%%%%%%%%%%
%%% Colors %
%%%%%%%%%%%%

\definecolor{blue}{RGB}{41,5,195}

%%%%%%%%%%%%%
%%% Figures %
%%%%%%%%%%%%%

\NewTotalTCBox
  {\figurewrapper}
  {mO{}m}
  { enhanced,
    center title,
    coltitle=black,
    fonttitle=\scriptsize,
    frame hidden,
    sharp corners,
    title={#1},
    #2}
  {#3}

%%%%%%%%%%
%%% Font %
%%%%%%%%%%

\usefonttheme{professionalfonts}
\renewcommand\familydefault{\rmdefault}

%%%%%%%%%%
%%% TikZ %
%%%%%%%%%%

\usetikzlibrary{calc,decorations.pathmorphing,positioning}

%%%%%%%%%%%%%%%%
%%% Title Page %
%%%%%%%%%%%%%%%%

\title{\worktitle}
\subtitle{\worksubtitle}
\author{Átila Gama Silva}
\date{\today}
\titlegraphic{\includegraphics[height=1cm]{../ifba-irece-logo.png}}

%%%%%%%%%%%%
% Document %
%%%%%%%%%%%%

\begin{document}
  %%%%%%%%%%%
  % Lengths %
  %%%%%%%%%%%

  %%%%%%%%%%%%%%
  %% Half Part %
  %%%%%%%%%%%%%%

  \newlength{\halfpart}
  \setlength{\halfpart}{.5\textwidth - .5em}

  %%%%%%%%%%%%%%%%%%%%%%%%%%%%%%%%%
  %% Text Width Plus Two em Parts %
  %%%%%%%%%%%%%%%%%%%%%%%%%%%%%%%%%

  \newlength{\halftextwidthplustwoempart}
  \setlength{\halftextwidthplustwoempart}{.5\textwidth + 2em}
  \newlength{\halftextwidthplustwoemcounterpart}
  \setlength{\halftextwidthplustwoemcounterpart}{.5\textwidth - 3em}

  %%%%%%%%%%%%%%%%%%%%%%%%%%%%%%%%%%
  %% Text Width Plus Five em Parts %
  %%%%%%%%%%%%%%%%%%%%%%%%%%%%%%%%%%

  \newlength{\halftextwidthplusfiveempart}
  \setlength{\halftextwidthplusfiveempart}{.5\textwidth + 5em}
  \newlength{\halftextwidthplusfiveemcounterpart}
  \setlength{\halftextwidthplusfiveemcounterpart}{.5\textwidth - 6em}

  %%%%%%%%%%%%%%
  % Title Page %
  %%%%%%%%%%%%%%

  \begin{frame}\titlepage\end{frame}

  %%%%%%%%%%%%%%%
  % Motivations %
  %%%%%%%%%%%%%%%

  \begin{frame}{Motivações}
    \begin{itemize}
      \item Dificuldades ao estudar estilos de formatação em diversas linguagens
            \begin{itemize}
              \item Estilos convencionais
                    \begin{itemize}
                      \item Recorrência a diferentes \textit{prettyprinters}
                      \item Diferentes configurações e níveis de suporte
                    \end{itemize}
              \item Estilos não convencionais
                    \begin{itemize}
                      \item Aplicação manual inevitável
                      \item Consumo de tempo e esforço
                    \end{itemize}
            \end{itemize}
    \end{itemize}
  \end{frame}

  %%%%%%%%%%%
  % Problem %
  %%%%%%%%%%%

  \begin{frame}{Problemática}
    \begin{itemize}
      \item Em geral, os formatadores de código
            \begin{itemize}
              \item São restritos a
                    \begin{itemize}
                      \item Uma linguagem específica
                      \item Uma família de linguagens de programação
                    \end{itemize}
              \item São limitados nas configurações de estilização
              \item Proporcionam pouca personalização
            \end{itemize}
    \end{itemize}
  \end{frame}

  %%%%%%%%%%%%%%%%%%%%%%
  % General Objectives %
  %%%%%%%%%%%%%%%%%%%%%%

  \begin{frame}{Objetivos Gerais}
    \begin{itemize}
      \item Desenvolver um software de linha de comando
      \item De natureza prototípica
      \item Para a formatação de código-fonte
      \item Tendo como objetivos
            \begin{itemize}
              \item Ser multilíngue
              \item Proporcionar a formatação personalizada
                    \begin{itemize}
                      \item Via a linguagem de consulta do \treesitter
                    \end{itemize}
            \end{itemize}
    \end{itemize}
  \end{frame}

  %%%%%%%%%%%%%%%%%%%%%%%
  % Specific Objectives %
  %%%%%%%%%%%%%%%%%%%%%%%

  \begin{frame}[fragile]{Objetivos Específicos}
    \begin{itemize}
      \item Desenvolver um algoritmo de formatação
            \begin{itemize}
              \item Fundamentado no \treesitter
            \end{itemize}
      \item Definir configurações de estilização para o predicado
            \codesnippetinline{set!}
      \item Estender os predicados da linguagem de consulta
            \begin{itemize}
              \item Proporcionando predicados basais para a formatação
            \end{itemize}
    \end{itemize}
  \end{frame}

  %%%%%%%%%%%%%%%%%%%%
  % Expected Results %
  %%%%%%%%%%%%%%%%%%%%

  \begin{frame}{Resultados Esperados}
    \begin{itemize}
      \item Abranger qualquer linguagem suportada pelo \treesitter
      \item Possibilitar procedimentos básicos de formatação
            \begin{itemize}
              \item Através dos predicados desenvolvidos
            \end{itemize}
    \end{itemize}
  \end{frame}

  %%%%%%%%%%%%%%%
  % Limitations %
  %%%%%%%%%%%%%%%

  \begin{frame}{Limitações}
    \begin{itemize}
      \item Desenvolvimento de predicados limitado a procedimentos básicos de
            formatação
      \item Ausência de mecanismos sofisticados de formatação
            \begin{itemize}
              \item E.g., formatação condicional
            \end{itemize}
    \end{itemize}
  \end{frame}

  %%%%%%%%%%%%%%%%%%%%%%%
  % The Code Formatting %
  %%%%%%%%%%%%%%%%%%%%%%%

  \begin{frame}{A Formatação de Código-Fonte}
    \begin{itemize}
      \item \textcite{oppen-1980-prettyprinting} apresentou um algoritmo
            de formatação multilíngue
            \begin{itemize}
              \item Baseado em anotações delimitando blocos
                    \begin{itemize}
                      \item Feitas por uma ferramenta intermediária
                    \end{itemize}
            \end{itemize}
      \item \textcite{yelland-2015-new} descreveu um algoritmo de otimização de
            layout do código
            \begin{itemize}
              \item Relativo a uma noção intuitiva de custo de layout
              \item Onde empregou-se os \textit{combinators}
                    \begin{itemize}
                      \item Funções geradoras descrevendo layouts alternativos
                            para o código
                    \end{itemize}
            \end{itemize}
    \end{itemize}
  \end{frame}

  %%%%%%%%%%%%%%%%%%%%%%%%%%
  % The Tree-Sitter System %
  %%%%%%%%%%%%%%%%%%%%%%%%%%

  \begin{frame}{O Sitema \treesitter}
    \begin{itemize}
      \item Sistema multilíngue de análise sintática
      \item Oferece uma linguagem de consulta declarativa
            \begin{itemize}
              \item Expressa padrões da árvore sintática
              \item Por meio de \textit{S-expressions}
              \item Busca correspondências
              \item Proporciona o uso --- e extensão --- de predicados
                    \begin{itemize}
                      \item Funções arbitrárias que filtram nós e realizam
                            verificações complexas
                    \end{itemize}
            \end{itemize}
    \end{itemize}
  \end{frame}

  %%%%%%%%%%%%%
  % Materials %
  %%%%%%%%%%%%%

  \begin{frame}{Materiais}
    \begin{itemize}
      \item Ecossistema Rust
      \item Sistema/biblioteca \treesitter \cite{tree-sitter-2023-tree}
      \item Ecossistema Neovim
    \end{itemize}
  \end{frame}

  %%%%%%%%%%%
  % Methods %
  %%%%%%%%%%%

  \begin{frame}{Métodos}
    \begin{itemize}
      \item Pesquisa experimental
            \begin{itemize}
              \item Explorando a aplicação do \treesitter
                    \begin{itemize}
                      \item Como base para o algoritmo de formatação
                    \end{itemize}
            \end{itemize}
      \item Estudo de caso
            \begin{itemize}
              \item Analisando a eficácia do software desenvolvido
            \end{itemize}
    \end{itemize}
  \end{frame}

  %%%%%%%%%
  % Usage %
  %%%%%%%%%

  \begin{frame}[fragile]{Usagem}
    \begin{itemize}
      \item \codesnippetinline{cook [-l LANG] -q QUERY [SRC]}
    \end{itemize}
  \end{frame}

  %%%%%%%%%%%%%%%%%%%%%%%%%%%%
  % The Formatting Algorithm %
  %%%%%%%%%%%%%%%%%%%%%%%%%%%%

  \begin{frame}{O Algoritmo de Formatação}
    \begin{center}
      \figurewrapper
        {Fluxograma do Algoritmo de Formatação}
        { \begin
            {tikzpicture}
            [node distance=1em,every node/.style={draw,inner sep=1em}]
            \tiny
            \node (init) {
              \begin
                {tikzpicture}
                [ every node/.style=
                    {align=center,draw,inner sep=.5em,text width=14em}]
                \node [draw=none] (label) {Inicialização};
                \node (I) [below=of label] {Analisa a árvore de sintaxe};
                \node (II) [below=of I] {Gera o objeto de consulta};
                \node (III) [below=of II] {Instancia a estrutura editora};
                \node (IV) [below=of III] {Inicializa o objeto de configurações};
                \node (V) [below=of IV] {Deriva a estrutura de correspondências};
                \draw [->] (I) -- (II);
                \draw [->] (II) -- (III);
                \draw [->] (III) -- (IV);
                \draw [->] (IV) -- (V);
              \end{tikzpicture}
            };
            \node (fmt) [right=of init] {
              \begin
                {tikzpicture}
                [ every node/.style=
                    {align=center,draw,inner sep=.5em,text width=14em}]
                \node [draw=none,text width=16em] (label) {Formatação};
                \path
                  let \p1=(label.south west)
                  in coordinate (I-coord) at ($ (\x1,\y1-2em) $);
                \node (I) at (I-coord) {Itera sobre os padrões};
                \node (II) [below=of I] {Define o escopo de atuação};
                \node
                  (III)
                  [below=of II]
                  {Itera sobre as coleções\\de nós capturados};
                \node (IV) [below=of III] {Aplica as configurações};
                \node (V) [below=of IV] {Aplica os predicados gerais};
                \node (VI) [below=of V] {Redefine as\\configurações locais};
                \node
                  (VII)
                  [below=of VI]
                  {Interrompe a iteração de\\coleções se o padrão\\é enraizado};
                \draw [->] (I) -- (II);
                \draw [->] (II) -- (III);
                \draw [->] (III) -- (IV);
                \draw [->] (IV) -- (V);
                \draw [->] (V) -- (VI);
                \draw [->] (VI) -- (VII);
                \draw [->] (VII.east) -- ++(1em,0) |- (III.east);
                \draw [->] (VII.east) -- ++(2em,0) |- (I.east);
              \end{tikzpicture}
            };
            \draw [->] (init) -- (fmt);
            \node (ret) [right=of fmt] {
              \begin
                {tikzpicture}
                [ every node/.style=
                    {align=center,draw,inner sep=.5em,text width=16em}]
                \node [draw=none] (label) {Retorno};
                \node
                  (label)
                  [below=of label]
                  {Retorna o código-fonte formatado};
              \end{tikzpicture}
            };
            \draw [->] (fmt) -- (ret);
          \end{tikzpicture}}
    \end{center}
  \end{frame}

  %%%%%%%%%%%%%%%%%%%%%%%%%%%%%
  % The Formatting Guidelines %
  %%%%%%%%%%%%%%%%%%%%%%%%%%%%%

  \begin{frame}[fragile]{As Diretrizes de Formatação}
    \begin{itemize}
      \item Especificadas no arquivo submetido via \codesnippetinline{-q QUERY}
      \item Para realizar um procedimento de formatação, é necessário
            \begin{itemize}
              \item Definir um padrão de correspondência
                    \begin{itemize}
                      \item Delimitando o escopo de operação
                    \end{itemize}
              \item Capturar nós que
                    \begin{itemize}
                      \item Sejam alvos do procedimento
                      \item Auxiliarão nas operações
                    \end{itemize}
              \item Opcionalmente, aplicar configurações --- via o predicado
                    \codesnippetinline{set!}
              \item Aplicar os predicados estendidos pelo \witchcooking
            \end{itemize}
    \end{itemize}
  \end{frame}

  %%%%%%%%%%%%%%%%%%%%%%
  % The Configurations %
  %%%%%%%%%%%%%%%%%%%%%%

  \begin{frame}[fragile]{As Configurações}
    \begin{itemize}
      \item \codesnippetinline{indent-rule}
            \begin{itemize}
              \item Define a regra de indentação para um nó
              \item Seu valor pode ser
                    \begin{itemize}
                      \item Um inteiro não negativo --- sem sinal
                      \item Um inteiro positivo --- com sinal
                      \item Um inteiro negativo
                    \end{itemize}
            \end{itemize}
      \item \codesnippetinline{indent-style}
            \begin{itemize}
              \item Define a string usada para indentar
              \item Pode ser de escopo local ou global
              \item Não aplicável a nós
            \end{itemize}
    \end{itemize}
  \end{frame}

  %%%%%%%%%%%%%%%%%%
  % The Predicates %
  %%%%%%%%%%%%%%%%%%

  %%%%%%%%%%%
  %% space! %
  %%%%%%%%%%%

  \begin{frame}[fragile]{Os Predicados}{\textit{space!}}
    \begin{minipage}[t]{\halftextwidthplustwoemcounterpart}
      \begin{codesnippet}[rust]{Função Com Bloco Aglomerado}
        fn x_plus_y() -> u32 {
          let x = 5; let y = 11;

          x + y
        }
      \end{codesnippet}
    \end{minipage}
    \hfill
    \begin{minipage}[t]{\halftextwidthplustwoempart}
      \begin
        {codesnippet}%
        [scheme]%
        {Consulta para Desaglomerar Bloco}
        (function_item
          body: (block (_) @item . (_) @next)
          (#space! "\n" 2 @item @next))
      \end{codesnippet}
    \end{minipage}

    \begin
      {codesnippet}%
      [rust]%
      {Função Com Bloco Desaglomerado}
      fn x_plus_y() -> u32 {
        let x = 5;
      let y = 11;

        x + y
      }
    \end{codesnippet}
  \end{frame}

  %%%%%%%%%%%%
  %% indent! %
  %%%%%%%%%%%%

  \begin{frame}[fragile]{Os Predicados}{\textit{indent!}}
    \begin{minipage}[t]{\halftextwidthplustwoemcounterpart}
      \begin{codesnippet}[rust]{Função Com Bloco Aglomerado}
        fn x_plus_y() -> u32 {
          let x = 5; let y = 11;

          x + y
        }
      \end{codesnippet}
    \end{minipage}
    \hfill
    \begin{minipage}[t]{\halftextwidthplustwoempart}
      \begin
        {codesnippet}%
        [scheme]%
        {Consulta para Desaglomerar Bloco Com Indentação}
        (#set! indent-style "  ")

        (function_item
          body: (block (_) @item . (_) @next)
          (#space! "\n" 2 @item @next)
          (#set! @next indent-rule "+1")
          (#indent! @next))
      \end{codesnippet}
    \end{minipage}

    \begin
      {codesnippet}%
      [rust]%
      {Função Com Bloco Desaglomerado e Indentação Apropriada}
      fn x_plus_y() -> u32 {
        let x = 5;
        let y = 11;

        x + y
      }
    \end{codesnippet}
  \end{frame}

  %%%%%%%%%%%%%%%%%%%
  %% indent-offset! %
  %%%%%%%%%%%%%%%%%%%

  \begin{frame}[fragile]{Os Predicados}{\textit{indent-offset!}}
    \begin{minipage}[t]{\halftextwidthplusfiveemcounterpart}
      \begin{codesnippet}[rust]{Função Compactada}
        fn foo() {bar()}
      \end{codesnippet}
    \end{minipage}
    \hfill
    \begin{minipage}[t]{\halftextwidthplusfiveempart}
      \begin
        {codesnippet}%
        [scheme]%
        {Consulta para Formatar uma Função Conforme o Estilo \textit{1TBS}}
        (#set! indent-style "    ")

        ( (function_item
            body: (block (_) @item "}" @close)) @fn
          (#set! @item indent-rule "+1")
          (#indent-offset! @close @fn)
          (#indent! @item @close))
      \end{codesnippet}
    \end{minipage}

    \begin
      {codesnippet}%
      [rust]%
      {Função Formatada Conforme o Estilo \textit{1TBS}}
      fn foo() {
          bar()
      }
    \end{codesnippet}
  \end{frame}

  %%%%%%%%%%%%%%%%%%%%%%%%
  % Node Synchronization %
  %%%%%%%%%%%%%%%%%%%%%%%%

  \begin{frame}[fragile]{A Sincronização de Nós}
    \begin{minipage}[t]{\halftextwidthplusfiveempart}
      \begin{codesnippet}[rust]{Funções Aninhadas}
        fn foo() {fn bar() {baz()}}
      \end{codesnippet}

      \begin
        {codesnippet}%
        [scheme]%
        {Consulta para Formatar Funções Conforme uma Variante do \textit{1TBS}}
        (#set! indent-style "  ")

        ( (function_item
            body: (block (_) @item "}" @close)) @fn
          (#set! @item indent-rule "+1")
          (#indent-offset! @close @fn)
          (#indent! @item @close))
      \end{codesnippet}
    \end{minipage}
    \hfill
    \begin{minipage}[t]{\halftextwidthplusfiveemcounterpart}
      \begin
        {codesnippet}%
        [rust]%
        {Funções Aninhadas Conforme uma Variante do \textit{1TBS}}
        fn foo() {
          fn bar() {
            baz()
          }
        }
      \end{codesnippet}
      \hfil
      \begin
        {codesnippet}%
        [rust]%
        {Funções Aninhadas Mal Formatadas}
        fn foo() {
          fn bar() {
          baz()
        }
      \end{codesnippet}
    \end{minipage}
  \end{frame}

  %%%%%%%%%%%%%%
  % Conclusion %
  %%%%%%%%%%%%%%

  \begin{frame}[fragile]{Conclusão}
    \begin{itemize}
      \item O objetivo geral deste trabalho foi atentido --- desenvolver o
            \witchcooking{}
      \item O software desenvolvido atendeu aos objetivos de
            \begin{itemize}
              \item Abranger uma gama de linguagens de programação
              \item Proporcionar a formatação personalizada via a linguagem de
                    consulta do \treesitter
                    \begin{itemize}
                      \alert{\item Não funcional para cenários realistas}
                    \end{itemize}
            \end{itemize}
    \end{itemize}
  \end{frame}

  %%%%%%%%%%%%%%%%%
  % Contributions %
  %%%%%%%%%%%%%%%%%

  \begin{frame}{Contribuições}
    \begin{itemize}
      \item O \witchcooking
            \begin{itemize}
              \item Formata diversas linguagens através do \treesitter
              \item Proporciona predicados para dirigir a formatação
              \item Oferece maior controle ao usuário
              \alert
                { \item Exige conhecimento
                    \begin{itemize}
                      \item Da sintaxe em questão
                      \item Da linguagem de consulta
                    \end{itemize}}
            \end{itemize}
    \end{itemize}
  \end{frame}

  %%%%%%%%%%%%%%%%
  % Future Works %
  %%%%%%%%%%%%%%%%

  \begin{frame}[fragile]{Trabalhos Futuros}
    \begin{itemize}
      \item Desenvolver um algoritmo dedicado à sincronização de nós
      \item Aprimorar o predicado \codesnippetinline{indent!}
      \item Dinamizar a formatação
            \begin{itemize}
              \item Formatar conforme \textit{CPL}
              \item Cálculo layout otimizado
            \end{itemize}
      \item Disponibilizar diretrizes de formatação
    \end{itemize}
  \end{frame}

  %%%%%%%%%%%%%%%%
  % Bibliography %
  %%%%%%%%%%%%%%%%

  \begin{frame}[allowframebreaks]{Referências}\printbibliography\end{frame}
\end{document}